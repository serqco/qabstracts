% appendix, to be included in the ArXiv version, but not the journal version.

\appendix

\renewcommand{\thefigure}{A.\arabic{figure}}
\def\topfraction{1.0}
\def\bottomfraction{1.0}


This appendix contains figures that provide additional detail for information
that is only summarized in the article proper.


\Plot{boxplots_words}{%
  Typical abstracts are 190 to 280 words long. 
  Structured abstracts tend to be longer than unstructured ones.}
\Plot{boxplots_sentences}{%
	Typical abstracts are 8 to 14 sentences long.
	The values for structured abstracts are inflated by the section headings,
	which each count as a sentence if they terminate in a colon.}
\Plot{boxplots_avg_wordlength}{%
	The average number of characters in a word is similar in all subgroups.}

\Plot{ab_topicstructure_freqs_empir_structured}{%
  The frequency of different trains-of-thought in the abstract for empirical articles
  using a structured abstract format.
  The "Method:" etc. subheadings are ignored for structured abstracts in all three plots.}


\Plot{boxplots_icount}{%
  A majority of abstracts has one or more ``informativeness gaps´´.
  The problem tends to be smaller in structured abstracts.}
\Plot{boxplots_ucount}{%
  Some abstracts have understandability gaps where an important concept is not explained at all.
  !!!}

\Plotwide{box_xletgroups_topicfractions}{%
	Per-topic distribution of the amount of space used for that topic.\\
	The plots in each group show these different subsets of abstracts:
	all, structured, non-structured, design, empirical, EMSE, ICSE, IST, TOSEM, TSE}
\Plotwide{nonzerofractionbar_xletgroups_missinginfofractions}{%
	What fraction of abstracts has the following gaps?
	Only announcing (instead of describing) methods, results, conclusions, possible future research.
	Not reporting simple details (``informativeness gaps´´), not explaining key terms (``understandability gaps´´).
	Announcing is generally not rare, in particular for results, and tends to be less pronounced
	at IST. 
	Missing to include detail once is dominant.
	Doing this several times is frequent and worse for design articles.
	Not explaining key terms is not rare and worst at EMSE.\\
	The plots in each group show these different subsets of abstracts:
	all, structured, non-structured, design, empirical, EMSE, ICSE, IST, TOSEM, TSE}
\Plot{boxplots_fraction_introduction}{%
	Authors invest about a third of the space into the introduction:
	background/context information and perhaps explaining a research gap.
	Structured abstracts tend to have less of this???}
\Plot{boxplots_fraction_conclusion}{%
	Many abstracts offer no conclusion, i.e., no explicit generalization of the immediate result.
	This is worst at ICSE (with over 75\%)
	and best at IST, which is the only venue where a majority of articles formulates a generalization.}

\Plot{lowess_gaps_by_fracintro}{%
	This plot checks the expectation that spending much space on background
	will correlate with more i and u gaps.
	This is not the case.
	Each dot is one coding of an abstract, 
	the line is a locally weighted linear regression.}
