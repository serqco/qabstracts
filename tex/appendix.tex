% appendix, to be included in the ArXiv version, but not the journal version.

\clearpage
\appendix

\renewcommand{\thefigure}{A.\arabic{figure}}
\def\topfraction{1.0}
\def\bottomfraction{1.0}


This appendix contains figures that provide additional detail for information
that is only summarized in the article proper.


\Plot{boxplots_words}{%
  Typical abstracts are 190 to 280 words long. 
  Structured abstracts tend to be longer than unstructured ones.}
\Plot{boxplots_sentences}{%
	Typical abstracts are 8 to 14 sentences long.
	The values for structured abstracts are inflated by the section headings,
	which each count as a sentence if they terminate in a colon.}
\Plot{boxplots_avg_wordlength}{%
	The average number of characters in a word is similar in all subgroups.}

\Plot{ab_topicstructure_freqs_empir_structured}{%
  The frequency of different trains-of-thought in the abstract for empirical articles
  using a structured abstract format.}


\Plot{boxplots_icount}{%
  Frequency of informativeness gaps: A majority of abstracts has one or more of them.
  The problem tends to be smaller in structured abstracts.}
\Plot{boxplots_ucount}{%
  Frequency of understandability gaps: 
  Some abstracts have such spots where an important concept is not explained at all.}

\Plot{boxplots_fraction_introduction}{%
	Authors tend to invest a lot of the abstract's space (about a third) into the introduction:
	background/context information and perhaps explaining a research gap.
	This is less pronounced in structured abstracts.}
\Plot{boxplots_fraction_conclusion}{%
    Fraction of the abstract's space that gets invested into the conclusion.
	Discussion: The conclusion is (or could and should be) the key takeaway for the reader, so 
	a good abstract should probably spend enough space (perhaps about 10\%?) on the conclusion.
	As we see here, very many articles offer less.
	Many abstracts offer no conclusion at all.
	This is worst at ICSE (with over 75\%)
	and best at IST, which is the only venue where a majority of articles formulates a generalization.}

\Plot{lowess_gaps_by_fracintro}{%
	This plot explores the expectation that spending much space on background
	will correlate with more i and u gaps.
	This does not appear to be the case.
	Each dot is one coding of an abstract, 
	the line is a locally weighted linear regression.}
