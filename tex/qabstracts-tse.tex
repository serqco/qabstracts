% Article about SE abstracts quality (qabstracts) for IEEE Trans. on Software Engineering
\documentclass[10pt,journal,compsoc]{IEEEtran}

\usepackage{cite}
%\usepackage[caption=false,font=normalsize,labelfont=sf,textfont=sf]{subfig}
%\usepackage{stfloats}
\usepackage{url}
%\usepackage{verbatim}
\usepackage{graphicx}
\graphicspath{ {../img} {.} }
\usepackage{balance}
\hyphenation{semi-conduc-tor IEEE-Xplore}

\newcommand{\Plot}[2]{%
	\begin{figure}[!t]%
		\centering\includegraphics[width=\columnwidth]{#1}%
		\vspace{-4mm}\caption{#2}\label{#1}%
	\end{figure}}

\newcommand{\Plotwide}[2]{%
	\begin{figure*}%
		\centering\includegraphics[width=\textwidth]{#1}%
		\vspace{-4mm}\caption{#2}\label{#1}%
    \end{figure*}}

\newcommand{\Todo}[1]{\bgroup\bf\Large #1\egroup}
\newcommand{\Quote}[1]{\bgroup\it #1\egroup}


\begin{document}
	
\title{How (Not) To Write a\\Software Engineering Abstract}

\author{Annie Author1, Bert Author2
\thanks{A. Author1 is with A University, Btown, Ccountry}
\thanks{B. Author2 is with D University, Etown, Fcountry}}

\markboth{IEEE Transactions on Software Engineering}%
{How (Not) To Write a Software Engineering Abstract}

\maketitle

\begin{abstract}  % 100 to 200 words are allowed (but not enforced)
\emph{Background:}
Abstracts are a very valuable element in a software engineering research article,
but not all abstracts are as informative as they could be.
\emph{Objective:} 
Characterize the structure of abstracts in high-quality venues, 
observe and quantify deficiencies, 
suggest guidelines for writing informative abstracts.
\emph{Methods:}
Use open coding to derive concepts that explain relevant properties of abstracts
and identify the archetypical structure of abstracts.
Use quantitative content analysis to objectively characterize abstract structure
over a sample of 500!!! abstracts from five presumably high-quality venues.
Use exploratory data analysis to find recurring issues in abstracts.
Compare the prototypical structure to actual structures to derive
guidelines for producing informative abstracts.
\emph{Results:}
!!!
\emph{Conclusions:}
(1)~Even in top venues, many abstracts are far from ideal.
(2)~Structured abstracts tend to be better than unstructured ones, 
but (3)~artifact-centric works need a different structured format.
(4)~The community should start requiring conclusions that generalize, 
which currently are often missing.
\end{abstract}

\begin{IEEEkeywords}
!!!.
\end{IEEEkeywords}


%========================================================================
\section{Introduction}
\IEEEPARstart{A}{lthough} the abstract is a super important part of any research article\Todo{reference!},
when reading an abstract in software engineering, even in a presumably top-quality venue,
we often feel it is lacking important information or find it difficult to understand at all.

The present article aims at substantiating this impression.


%Ideally, an abstract should 
%situate the work within software engineering,
%motivate and state the issue addressed by it,
%describe the methods applied,
%report empirical results,
%and formulate a useful take-home message.
%With the possible exception of discussing related work,
%this is much the same agenda as for the article itself,
%just with much less detail;
%getting it right is no easy task.


\subsection{Research Questions}

% we use the numbers only in this subsection and the next;
% everywhere else we repeat the question or paraphrase it.
We ask several questions:
\begin{enumerate}
	\item What does a typical well-written abstract look like?
	\item Which deficiencies occur? 
	\item How often?
	\item Do structured abstracts have better quality than unstructured ones?
	\item How should software engineering abstracts be written?
\end{enumerate}
Of these, numbers 1 and 2 should be considered exploratory,
number 4 is hypothesis-driven (we expect a yes), and
the answer to number 5 we consider a conclusion from the answers to the other four. 


\subsection{Research Approach}

No firm expectations regarding questions 1 and 2 exist for engineering articles,
so qualitative methods will have to be used for them:
We start our research by open coding \cite{StrCor90} in order to derive a vocabulary (a set of concepts or codes)
by which the nature of a particular abstract can be characterized.

At the same time, we intend to convince even people who are skeptical of qualitative research
regarding our answers to questions 3, 4, and 5 and regarding the need to improve
the quality of software engineering abstracts.
Therefore, we continue the research by performing a nearly objective 
quantitative content analysis for number 3
and apply an elaborate seven-step approach that even allows for quantifying the
remaining amount of subjectivity we could not eliminate.

The final stage is the statistical evaluation of the content analysis data,
which is again exploratory in principle, but actually straightforward in both
the questions it asks and the statistical methdods it uses for answering them.


\subsection{Research Contributions}

Our contributions correspond to the research questions as follows:
\begin{enumerate}
	\item As for the structure of well-written abstracts, we present an "abstract archetype"
	  that describes fixed parts of the structure and degrees of freedom. 
	\item We describe and discuss !!! types of deficiency 
	\item We quantify the frequency of each deficiency for the entire sample of abstracts
	  we looked at as well as for 9 different subgroups of interest.
	\item We present convincing data that structured abstracts tend to be better 
	  in several respects.
	\item We provide data-based how-to instructions for abstracts writing for authors
	  as well as guidance for editors and conference organizers.
	  Software Engineering works should use structured abstracts, but
	  need a different and more flexible template than what is used so far.
\end{enumerate}


%========================================================================
\section{Related Work}

There is a considerable literature on research abstracts across disciplines
and we will not attempt to summarize it here.
Instead, we only provide examples of the different major perspectives of those studies
and otherwise focus on what has been done in the software engineering domain.


\subsection{Abstract Structure}

Swales introduced "genre analysis" as a means for teaching academic reading and writing
especially to non-native speakers in 1990 \cite{Swales90}.
Genres are text types and genre analysis means deconstructing the composition
of texts (of that text type) in terms that help understand the elements in
terms of their syntactical structure, content, role, and interrelationships
(e.g. their relative position within the whole).

For our purposes here, the most relevant idea from genre analysis is the notion
of "moves", which are, roughly speaking, the building blocks used by the writers
for making their overall point.
Several studies have looked at the move structure of research abstracts
in different fields such as 
applied linguistics \cite{DosSantos96} or
protozoology \cite{CroOpp06}.
Despite the differences of research content, they find very similar move structures.
For instance, \cite{CroOpp06} formulates
\begin{quote}
	move 1 situates the research within the scientific community;\\
	move 2 introduces the research by either describing the main features of the
    research or presenting its purpose;\\
	move 3 describes the methodology;\\
	move 4 states the results; and\\
    move 5 draws conclusions or suggests practical applications.
\end{quote}
We found a similar structure (and use it with some extensions)
for empirical works in software engineering, 
but abstracts of artifact-centric works (tool building) do not fit this model and need
an extended one.


\subsection{Abstract Quality}

Once a "good" structure for abstracts is known, it provides an approach for
discussing the quality of specific actual abstracts.
For instance, \cite{CroOpp06}, where quality assessment is not a main goal,
nevertheless finds that one third of the 12 abstracts studied
is lacking move 2 (stating a purpose).

Several studies have, however, put quality assessment at their center,
most often in subfields of the biomedical domain.
Many such studies comprise articles of a homogeneous nature:
all randomized controlled trials (controlled experiments).
This allows formulating very specific expectations what information should be
presented in an abstract and allows performing the analysis in checklist fashion.
In this manner,
\begin{itemize}
  \item \cite{DupKhoLeb03} used a 30-item checklist on 197 abstracts in clinical dermatology
     for computing a 0-to-1 score and found mean scores between 0.64 and 0.78 for their various subgroups.
  \item \cite{ShaHar06} used a 29-item checklist on 100 abstracts in dental medicine
	 and found a mean score of only 0.54.
     % suggests an 8-move structure for structured abstracts: 
     % objective, design, setting, patients, interventions, outcome measure, results, conclusion.
     % per-move deficit rates are not reported.
  \item \cite{} used a !!!-item checklist on !!! abstracts in !!!field
	 and found !!!\% of them to have deficiencies,
\end{itemize}


\subsection{Structured vs.\ Unstructured Abstracts}

In many cases, and in particular in those parts of the biomedical domain 
where controlled experiments are the norm, quality-oriented studies do not study
quality in general.
For instance neither \cite{DupKhoLeb03} nor \cite{ShaHar06} reports
which of their items are missing most frequently.
Rather, their research question is the relative quality of structured abstracts
versus unstructured ones. 
And in almost all of those cases, including \cite{DupKhoLeb03} and \cite{ShaHar06},
the answer is: structured abstracts have fewer quality issues.

Such research can be highly influential.
For instance, \cite{HarSydBlu96} is cited in the CONSORTS report \cite{MohHopSch12},
which defines reporting guidelines for controlled experiments and has about
10000 citations.

!!!Budgen work!!!


%========================================================================
\section{Methods}


\subsection{Overview}

\noindent
This is a mixed-method study, consisting of three nominally sequential structural parts
(Prestudy1, Prestudy2, Fullstudy) and
four overlapping and partially iterative semantic elements
(codebook development, quantitative content analysis,
statistical evaluation, interpretation).

Overall, we consider our study to be a qualitative one, but note that the middle two elements,
quantitative content analysis and statistical evaluation,
are compatible with a positivist epistemology (with "objective" results) so that 
the final interpretation is done on a solid quantitative foundation.

Prestudy1 was for initial codebook development.
Prestudy2 was for codebook refinement,
coding procedure development,
and coder training.
Fullstudy was for all other elements.
See Figure~\ref{qabstracts_timeline} for a timeline of the lengthier aspects of the study.

\Plot{qabstracts_timeline}{%
	When the longer parts and elements of the study happened. (VEEERY PRELIMINARY!!!)}


\subsection{Data Availability}

\noindent
Full detail, including detailed execution history, available on GitHub:
Codebook, handling procedure, coded abstracts, scripts for automation, scripts for stats and plots.


\subsection{Codebook Development}

\noindent
Open coding in GTM fashion (grounding, constant comparison, concept memos).
Initially structured abstract (describe!); then discovered archetype and extended set of codes; lots of clarification during coder training.
49!!! codes; see codebook for details.
Codes for key parts are subtle ( e.g. gap must lead "directly" to objective).


\subsection{Coding Rules}

\noindent
by sentence; prefer single codes; consider only 1 sentence forward context (like readers); be friendly


\subsection{Sample}

\noindent
Presumably-top-quality venues EMSE, ICSE TR, TOSEM, TSE,
plus IST for structured abstracts.
Random sample from 2022 population.
100 each, but TOSEM smaller.


\subsection{Quantitative Content Analysis and Inter-Coder Disagreements}

\noindent
Extract abstracts from PDFs; automatically split by sentence; manually add codes in {{}};
code each abstract twice; script checks compatibility; coder B reconsiders; coder A reconsiders;
A and B discuss remainder and resolve or add -ignorediff.

Split disagreements into simple mistakes (and resolve those) and actual interpretation differences
(limits of objectivity; actually usually quality issues, see below).


\subsection{Statistical Evaluation}

\noindent
Plot-centric; straightforward; no bothering with confidence intervals.

Verbal terms for frequencies: 
rare (less than 5\%),
not rare (5-20\%),
common (20-35\%),
frequent (35-50\%),
dominant (over 50\%).


\subsection{Interpretation}

\noindent
Driven by research interest; grounded as far as possible.
Makes use of impressions not captured in the data.
Choice of examples reflects those.


%========================================================================
\section{Results}

\subsection{Design Articles vs. Empirical Articles}

\noindent
Very different character depending on main contribution.
Design articles have empirical studies, too.
Empirical articles may describe auxiliary or secondary artifacts, too.


\subsection{The Abstracts Archtetype}

\noindent



\subsection{How Not To: Missing Elements}

\noindent
Key elements (background, objective, (design), method, result, conclusion); helpful elements (summary, gap)


\subsection{How Not To: Convoluted Trains of Thought}

\noindent
Topicstructure sequences; Archetype violations.


\subsection{How Not To: Uninformative Formulations}

\noindent
Informativeness gaps; Announcements.


\subsection{How Not To: Undefined Inportant Terms}

\noindent
Understandability gaps


\subsection{How Not To: Ambiguous Formulations}

\noindent
Ignorediffs


\subsection{How-Not-To Summary}

\noindent
Good quality means: Follows archetype, no announcement, no gap.
Should be over 90\% of articles.


\subsection{How To: Structured Abstracts are more orderly}

\noindent
Re-analysis of "Convoluted Trains of Thought". 




%========================================================================
\section{Limitations and Threats to Validity}

\noindent
Internal: careful procedure, internal validity issues unlikely.
Construct: cannot measure informativeness; cannot measure understandability;
our proxy measures are less informative and only 90\% valid.
Generalizability: 
results should generalize to neighboring years in same venues;
results not better in second-tier venues?


%========================================================================
\section{Conclusions}


\subsection{...}
\noindent


\subsection{How To: Guidelines for Well-Written Abstracts}

\noindent
Authors: Follow archetype; write a structured abstract; consider a 'gap'; pack all information you can; formulate a take-home message (conclusion)

Venues: Prescribe structured abstracts: Background (not Context); Design (if needed); Method/Methods; Results; or: Substudy 1/Substudy 2; Conclusion



%========================================================================
\appendix
\section{Basic statistics}

\Plot{boxplots_words}{%
  Typical abstracts are 190 to 280 words long. 
  Structured abstracts tend to be longer than unstructured ones, 
  although IST abstracts overall (including the design articles) do not.}
\Plot{boxplots_sentences}{%
	Typical abstracts are 8 to 14 sentences long.
	Structured abstracts tend to be longer than unstructured ones.
	IST requires structured abstracts, the other venues do not.}
\Plot{boxplots_icount}{%
  A majority of abstracts has one or more "informativeness gaps".
  The problem tends to be smaller in structured abstracts.}
\Plot{boxplots_ucount}{%
  Some abstracts have understandability gaps where an important concept is not explained at all.
  !!!}
\Plotwide{zerofractionbar_xletgroups_topicmissingfractions}{%
	How often is a topic not present at all in an abstract?\\
	The plots in each group show these different subsets of abstracts:
	all, structured, non-structured, design, empirical, EMSE, ICSE, IST, TOSEM, TSE}
\Plotwide{box_xletgroups_topicfractions}{%
	Per-topic distribution of the amount of space used for that topic.\\
	The plots in each group show these different subsets of abstracts:
	all, structured, non-structured, design, empirical, EMSE, ICSE, IST, TOSEM, TSE}
\Plotwide{nonzerofractionbar_xletgroups_missinginfofractions}{%
	What fraction of abstracts has the following gaps?
	Only announcing (instead of describing) methods, results, conclusions, possible future research.
	Not reporting simple details ("informativeness gaps"), not explaining key terms ("understandability gaps").
	Announcing is generally not rare, in particular for results, and tends to be less pronounced
	at IST. 
	Missing to include detail once is dominant.
	Doing this several times is frequent and worse for design articles.
	Not explaining key terms is not rare and worst at EMSE.\\
	The plots in each group show these different subsets of abstracts:
	all, structured, non-structured, design, empirical, EMSE, ICSE, IST, TOSEM, TSE}
\Plot{ab_topicstructure_freqs_design}{%
  The frequency of different trains-of-thought in the abstract for design articles.
  The label is a string of stretch-code characters:
  b-ackground, g-ap, o-bjective, d-esign, m-ethod, r-esult, c-onclusion, f-uture.}
\Plot{ab_topicstructure_freqs_empir}{%
	The frequency of different trains-of-thought in the abstract for empirical articles.
	Same characters as before, except that d cannot occur.}
\Plot{boxplots_fraction_introduction}{%
	Authors invest about a third of the space into the introduction:
	background/context information and perhaps explaining a research gap.
	Structured abstracts tend to have less of this???}
\Plot{boxplots_fraction_conclusion}{%
	Many abstracts offer no conclusion, i.e., no explicit generalization of the immediate result.
	This is worst at ICSE (with over 75\%)
	and best at IST, which is the only venue where a majority of articles formulates a generalization.}
\Plot{lowess_gaps_by_fracintro}{%
	This plot checks the expectation that spending much space on background
	will correlate with more i and u gaps.
	This is not the case.
	Each dot is one coding of an abstract, 
	the line is a locally weighted linear regression.}

%\begin{figure}[!t]
%\centering
%\includegraphics[width=2.5in]{fig1}
%\caption{This is the caption for one fig.}
%\label{fig1}
%\end{figure}


\subsection{Acknowledgments}
\noindent We thank Gesine Milde for cleansing the automatically extracted abstract texts.



%\begin{IEEEbiographynophoto}{Jane Doe}
%Biography text here without a photo.
%\end{IEEEbiographynophoto}

%\begin{IEEEbiography}[{\includegraphics[width=1in,height=1.25in,clip,keepaspectratio]{fig1.png}}]{IEEE Publications Technology Team}
%In this paragraph you can place your educational, professional background and research and other interests.\end{IEEEbiography}


\end{document}


