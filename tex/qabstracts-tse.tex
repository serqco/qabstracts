% Article about SE abstracts quality (qabstracts) for IEEE Trans. on Software Engineering
\documentclass[journal]{IEEEtran}

\usepackage{cite}
%\usepackage[caption=false,font=normalsize,labelfont=sf,textfont=sf]{subfig}
%\usepackage{stfloats}
\usepackage{url}
%\usepackage{verbatim}
\usepackage{graphicx}
\graphicspath{ {../img} {.} }
\usepackage{balance}
\hyphenation{semi-conduc-tor IEEE-Xplore}

\newcommand{\Plot}[2]{%
	\begin{figure}[!t]%
		\centering\includegraphics[width=\columnwidth]{#1}%
		\vspace{-6mm}\caption{#2}\label{#1}%
	\end{figure}}

\begin{document}
	
\title{How (not) to write a software engineering abstract}
\author{Annie Author1, Bert Author2
\thanks{A. Author1 is with A University, Btown, Ccountry}
\thanks{B. Author2 is with D University, Etown, Fcountry}}

\markboth{IEEE Transactions on Software Engineering}%
{How (Not) To Write a Software Engineering Abstract}

\maketitle

\begin{abstract}  % 100 to 200 words are allowed (but not enforced)
\emph{Background:}
Abstracts are a very valuable element in a software engineering research article,
but not all abstracts are as informative as they could be.
\emph{Objective:} 
Characterize the structure of abstracts in top venues, 
observe deficiencies, 
suggest guidelines for writing informative abstracts.
\emph{Methods:}
Use open coding to derive concepts that explaining relevant properties of abstracts
and to identify the prototypical structure of abstracts.
Use quantitative content analysis to objectively characterize abstract structure
over a sample of 500!!! abstracts from five high-quality venues.
Use exploratory data analysis to find recurring issues in abstracts.
Compare the prototypical structure to actual structures to derive
guidelines for producing informative abstracts.
\emph{Results:}
!!!
\emph{Conclusions:}
(1)~Even in top venues, many abstracts are far from ideal.
(2)~Structured abstracts tend to be better than unstructured ones, 
but (3)~artifact-centric works need a different structured format.
(4)~The community should start requiring conclusions that generalize, 
which currently are often missing.
\emph{Background:}

\end{abstract}

\begin{IEEEkeywords}
!!!.
\end{IEEEkeywords}


\section{Introduction}
\IEEEPARstart{W}{elcome} to the article.




\section{Section Two}

\noindent Welcome to the second section.



\appendix
\section{Basic statistics}

\Plot{boxplots_sentences}{%
  Typical abstracts are 8 to 14 sentences long.
  Structured abstracts tend to be longer than unstructured ones.
  IST requires structured abstracts, the other venues do not.}
\Plot{boxplots_words}{%
  Typical abstracts are 190 to 280 words long. 
  Structured abstracts tend to be longer than unstructured ones, 
  although IST abstracts overall (including the design articles) do not.}
\Plot{boxplots_fraction_introduction}{%
  Authors invest about a third of the space into the introduction:
  background/context information and perhaps explaining a research gap.
  Structured abstracts tend have less of this???}
\Plot{boxplots_fraction_conclusion}{%
  Many abstracts offer no generalization.
  This is worst at ICSE (with over 75\%)
  and best at IST, which is the only venue where a majority of articles formulates a generalization.}
\Plot{boxplots_icount}{%
  A majority of abstracts has one or more "informativeness gaps".
  The problem tends to be smaller in structured abstracts.}
\Plot{boxplots_ucount}{%
  Some abstracts have understandability gaps where an important concept is not explained at all.
  !!!}
\Plot{lowess_gaps_by_fracintro}{%
  This plot checks the expectation that spending much space on background
  will correlate with more i and u gaps.
  This is not the case.
  Each dot is one coding of an abstract, 
  the line is a locally weighted linear regression.}

%\begin{figure}[!t]
%\centering
%\includegraphics[width=2.5in]{fig1}
%\caption{This is the caption for one fig.}
%\label{fig1}
%\end{figure}


\subsection{Acknowledgments}
\noindent We thank Gesine Milde for cleansing the automatically extracted abstract texts.



%\begin{IEEEbiographynophoto}{Jane Doe}
%Biography text here without a photo.
%\end{IEEEbiographynophoto}

%\begin{IEEEbiography}[{\includegraphics[width=1in,height=1.25in,clip,keepaspectratio]{fig1.png}}]{IEEE Publications Technology Team}
%In this paragraph you can place your educational, professional background and research and other interests.\end{IEEEbiography}


\end{document}


